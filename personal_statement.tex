\documentclass[12pt]{article}

\title{Personal Statement}
\date{\today}
\author{Dario Villanueva}

\begin{document}

\maketitle

Intellectually, I have always been at a crossroads between art and computer science. I feel there is an artistic part of me that wants to express itself through the medium I know the most: The Computer. This programme will provide me with the tools and knowledge to explore and mature the artist in me, through the medium that feels most natural to me. I currently create visuals and music generatively with a computer, and I appreciate the immense possibilities that the computer has as a medium for artistic expression. However, the entry barrier to this medium remains unusually high. Anyone can pick up a guitar and make a sound; anyone can pick up a pencil and draw. Sadly, that is not the case with computers.

There is a performer in me that wants to explore the medium of live coding, but, more critically, the research topic I've chosen will enable countless others to do the same. By being able to research live coding as a storable and transferable medium, I feel I can pave the way for others to better understand and study this nascent art form. 

I have a vision of a community of artists and scientists working together to create music and visuals, maybe even competitively, learning from each others’ performances, learning the expert’s techniques, perhaps creating different “schools” or styles. This programme will enable me to build this platform, and once it is built, it will enable for the study (empirical or otherwise) of live coding performances. That is something that excites me greatly.

My academic background is mainly focused on the nature of intelligence and perception, and the design of systems that simulate and study them. My MA in Cognitive Science, encompassed subjects from both the realm of science (Artificial Intelligence, Computer Science) and the humanities (Psychology, Linguistics and Philosophy). This was no coincidence, but a conscious choice to take on an interdisciplinary route that has proven fruitful throughout my professional career.

This interdisciplinary programme meant that I was able to learn research techniques used in Psychology - which later became very useful in industry (A/B testing, data-driven design, etc), as well as complex mathematical methods, machine learning and Artificial Intelligence concepts. This humanistic approach has always helped me excel in my career, and that will undoubtedly prove useful for this research program.

Going further back into my pre-university academics, having completed the International Baccalaureate in a UWC institution has equipped me with a solid foundation for critical thinking, responsibility and creativity. Also it provided me with a world view that makes me an ambitious but humble and understanding individual

My extensive work experience means that I have the ability and knowledge required to complete and solve the technical challenges of this research program. My plan of building an online platform for users to like, enjoy and actually use is not an easy feat, and requires building a product that in normal circumstances would be a commercial endeavour, probably created by a team of several developers and designers. It is only through knowledge and experience in the industry that I feel confident in my capabilities to create a platform that artists can enjoy using, one which will gain traction to become both an artistic community and a tool for research.

Most of my previous work in industry has been as a Data Visualisation expert, which in a way is a form of expression and communication. I have learnt valuable lessons from trying to communicate to people certain ideas in a visual manner, lessons that will be vital for completing this programme. 

There are several long-term academic goals that I wish to achieve with the completion of this course. First and foremost I would like to contribute academically to live coding, increasing its recognition as an art form in its own right, but also enabling the scientific and arts community to look at these pieces in a new manner. I compare it to the advent of sheet music, which enabled repetition and the advent of the composer. I feel that creating a platform that encourages sharing and collaboration will increase the chances of discovering new talented artists and new ways for them to express themselves.

I would like to become more versed in live-coding myself, and creating a platform where pieces can be played back and analysed will help me (and countless others) in achieving this goal. If successful, my research will pave the way for others to formally research live coding in ways we can only think about right now, which I detail more clearly in my proposal. 

Finally, a return to an academic setting will provide me with a new challenge, growing my network, helping me meet new talented individuals, and expanding my horizons greatly both as an artist and a scientist. I feel that Goldsmiths’ focus on people will play a vital role in this, and I can but imagine the work I can achieve by being part of this amazing institution.

\end{document}
