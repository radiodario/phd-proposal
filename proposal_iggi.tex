\documentclass[12pt]{article}
\usepackage[sorting=none]{biblatex}
\PassOptionsToPackage{hyphens}{url}\usepackage{hyperref}

\bibliography{proposal}

\title{Research Propsal: Live coding performances in a massively-online game platform}
\date{\today}
\author{Darío Villanueva}

\begin{document}

\maketitle

\section*{Introduction}

Live coding is an art form. Current audiovisual performance languages~\cite{gibber, tidal,supercollider,livecodelab} put code at the artist's disposal to act as a bridge between the analog and the digital~\cite{textural-x}. These talented individuals create pieces that enthral and engage audiences across the world. However, these experiences exist momentarily, only whilst being executed by the artist. They are usually recorded as videos, which effectively ``freeze'' them into a format that doesn't easily afford rework, study or analysis.

Live coding is in its infancy, and the ``raster'' nature of the current mediums of storage means that the study of these performances is practically limited to manual work. While many artists divulge their pieces as source code files, these lack a key aspect: \emph{a representation of the process undergone by the artist while performing}. This is key, and relates to the first point of the TOPLAP manifesto: \emph{Give us access to the performer's mind, to the whole human instrument}~\cite{toplap-manifesto}

As such, live coding remains a form of art with a very high threshold for access. Authors have outlined techniques to improve and practice livecoding, in the form of games and competitive scenarios such as \emph{laptop battles}, \emph{Scrabble}, \emph{Tetris Challenge} etc~\cite{laptop-performance, live-coding-practice}.

I propose the creation of a platform that allows live coding performances to be stored and transmitted. Its objective is to encourage collaboration, play, study and practise. By establishing this online infrastructure, it will be possible to create different game modes and competitive scenarios, which I predict will attract more potential artists to this novel art form.


\section*{Research Questions}

\begin{itemize}

\item
Creating a technique to serialise, broadcast, store, visualise and reproduce live coding performances to an audience.

\item
Creating an online platform that leverages the aforementioned technique in which artists can perform, interact, collaborate and compete in different game modes and events.

\item
How will this newly afforded replayability and serialisability help lower the high entry barrier to live coding. Can an on-line platform help others learn how to livecode more effectively?

\item
Given an online competitive arena, can live coding become a game, perhaps a sport? Can this platform spark the creation of novel musical games?
\end{itemize}

\printbibliography

\end{document}
